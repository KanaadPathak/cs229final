\documentclass{article}
\usepackage{graphicx}
\usepackage{titlesec}
\usepackage{hyperref}
\usepackage{enumitem}
\usepackage{lmodern}
\usepackage{amsmath}
\usepackage{fancyhdr}
\usepackage{textcomp}
\usepackage{lmodern}% http://ctan.org/pkg/lm
\usepackage[table,x11names,svgnames]{xcolor}
\usepackage{soul}
\usepackage{parskip}
\usepackage{multirow}
\usepackage{array}
\usepackage{chngcntr}
\usepackage{afterpage}
\usepackage{tabularx}
\usepackage{float}
\usepackage{placeins}
\usepackage{tablefootnote}
\usepackage{microtype}
\usepackage{textcomp}
\usepackage{titlesec}
\usepackage{enumitem}
\usepackage{listings}
\usepackage{subcaption}
\usepackage{verbatim}
\usepackage[htt]{hyphenat}
\usepackage[letterpaper, portrait, margin=1.5in]{geometry}
\counterwithin{table}{section}
\counterwithin{figure}{section}

% Directives
\setlength\extrarowheight{5pt}

\setcounter{secnumdepth}{4}
\titleformat{\paragraph}
{\normalfont\normalsize\bfseries}{\theparagraph}{1em}{}

\begin{document}

\hrulefill \par
{\Large \textbf{Plant leaf classification}\par}
{\Large Proposal\par}
\hrulefill \par
{\bf Team members\par}
\begin{align*}
\mathrm{Albert\ Liu/SUID: 06167442}
  &&\href{mailto:albertpl@stanford.edu}{albertpl@stanford.edu}\\
\mathrm{Yangming\ Huang/SUID: 06168334}
  &&\href{mailto:yangming@stanford.edu@standford.edu}{yangming@stanford.edu}\\
\end{align*}

{\Large Introduction\par}
The number of plant species is estimated to be over 220000 in the world \cite{Charles13}. 
Automatic plant species recognition with image processing has gained increasing
interests recently. The main application are crop/weeds identification, plan
biology research and species tracking \cite{Pedro13}. 
Literature survey suggests that leave images are considered the most available and
effective attributes for such identification.  In our view, this problem is a
multi-class classification with relatively small training samples (raw images of
leaf).

The main goal of this project is to apply various algorithms and techniques we
learned in this class to solve this problem. First step is to use KNN on the
pre-extracted {shape, margin, texture} features to establish baseline. This is chosen
because most of existing works \cite{Charles13} \cite{Pedro13} recommend KNN and
the aforementioned features are readily available from the data sets.  Secondly
we will approach with a set of different models, including Linear SVM, Quadratic SVM,
Linear Discriminative Analysis and others. Then we compare the results of
these algorithms. Lastly we will explore to extract customer features from
the raw images and apply feature extraction/selection techniques. The intention is 
to figure out what combinations of features and algorithms give the best performance.

So far we have found the following data set
\begin{enumerate}
  \item UCI \cite{UCIDataSet} 40 species with 5 to 16 samples per species
  \item kaggle \cite{KaggleDataSet} 99 species with 16 samples per species
\end{enumerate}

The feature extraction techniques are discussed in \cite{Pedro13}.

\begin{thebibliography}{9}
\bibitem{Charles13}
Charles Mallah, James Cope, James Orwell. Plant Leaf Classification Using Probabilistic Integration of Shape, Texture and Margin Features. Signal Processing, Pattern Recognition and Applications, in press. 2013

\bibitem{Pedro13}
Evaluation of Features for Leaf Discrimination, Pedro F. B. Silva, Andre R.S. Marcal, Rubim M. Almeida da Silva (2013), Springer Lecture Notes in Computer Science, Vol. 7950, 197-204.

\bibitem{UCIDataSet}
https://archive.ics.uci.edu/ml/datasets/Leaf

\bibitem{KaggleDataSet}
https://www.kaggle.com/c/leaf-classification/data

\end{thebibliography}
\end{document}
