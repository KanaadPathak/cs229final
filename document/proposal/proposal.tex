\documentclass{article}
\usepackage{graphicx}
\usepackage{titlesec}
\usepackage{hyperref}
\usepackage{enumitem}
\usepackage{lmodern}
\usepackage{amsmath}
\usepackage{fancyhdr}
\usepackage{textcomp}
\usepackage{lmodern}% http://ctan.org/pkg/lm
\usepackage[table,x11names,svgnames]{xcolor}
\usepackage{soul}
\usepackage{parskip}
\usepackage{multirow}
\usepackage{array}
\usepackage{chngcntr}
\usepackage{afterpage}
\usepackage{tabularx}
\usepackage{float}
\usepackage{placeins}
\usepackage{tablefootnote}
\usepackage{microtype}
\usepackage{textcomp}
\usepackage{titlesec}
\usepackage{enumitem}
\usepackage{listings}
\usepackage{subcaption}
\usepackage{verbatim}
\usepackage[htt]{hyphenat}
\usepackage[letterpaper, portrait, margin=1.5in]{geometry}
\counterwithin{table}{section}
\counterwithin{figure}{section}

% Modify this to change who \Apple is (e.g. Olympic, Silver, Orion, Star ...)

% Directives
\pagestyle{fancy}
\setlength\extrarowheight{5pt}
\titlespacing*{\section}
{0pt}{3.5ex plus 1ex minus .2ex}{2.3ex plus .2ex}
\titlespacing*{\subsection}
{0pt}{2.0ex plus 1ex minus .2ex}{1.3ex plus .2ex}
\titlespacing*{\subsubsection}
{0pt}{1.0ex plus 1ex minus .2ex}{0.8ex plus .2ex}

% Copypasta to get labeled paragraphs
\setcounter{secnumdepth}{4}
\titleformat{\paragraph}
{\normalfont\normalsize\bfseries}{\theparagraph}{1em}{}
\titlespacing*{\paragraph}
{0pt}{3.25ex plus 1ex minus .2ex}{1.5ex plus .2ex}

\begin{document}

{\Large \textbf{Plant leaf classification}\par}
{\Large Proposal\par}
\hrulefill \par
{\bf Team members\par}
\begin{align*}
\mathrm{Albert\ Liu/SUID: 06167442}
  &&\href{mailto:albertpl@stanford.edu}{albertpl@stanford.edu}\\
\mathrm{Yangming\ Huang/SUID: 06168334}
  &&\href{mailto:yangming@stanford.edu@standford.edu}{yangming@stanford.edu}\\
\end{align*}

\section{Introduction}
Automatic plant species recognition with image processing has gained increasing
interests recently. The main application are crop/weeds identification, plan
biology research and species tracking \cite{Pedro13}. The number of plant
species is over 220000 and leave images are considered the most available and
effective attributes for such identification.  This problem can be seen as a
multi-class classification with relatively small training samples (raw image of
leaf).

The main goal of this project is to apply various algorithms and techniques we
learned in this class to solve this problem. First step is to use KNN on the
pre-extracted {shape, margin, texture} features to establish baseline. This is
because most of existing works \cite{Charles13} \cite{Pedro13} suggest KNN and
the aforementioned features are readily available from the data sets.  Secondly
we will approach with a set of models, including Linear SVM, Quadratic SVM,
Linear Discriminative Analysis and others. Then we compare the results of
different algorithms. Lastly we will explore to extract customer features from
the raw images and apply feature extraction/selection techniques.

The data set
\begin{enumerate}
  \item UCI \cite{UCIDataSet} 40 species with 5 to 16 samples per species
  \item kaggle \cite{KaggleDataSet} 99 species with 16 samples per species
\end{enumerate}

The feature extraction techniques are discussed in \cite{Pedro13}.

\begin{thebibliography}{9}
\bibitem{Charles13}
Charles Mallah, James Cope, James Orwell. Plant Leaf Classification Using Probabilistic Integration of Shape, Texture and Margin Features. Signal Processing, Pattern Recognition and Applications, in press. 2013

\bibitem{Pedro13}
Evaluation of Features for Leaf Discrimination, Pedro F. B. Silva, Andre R.S. Marcal, Rubim M. Almeida da Silva (2013), Springer Lecture Notes in Computer Science, Vol. 7950, 197-204.

\bibitem{UCIDataSet}
https://archive.ics.uci.edu/ml/datasets/Leaf

\bibitem{KaggleDataSet}
https://www.kaggle.com/c/leaf-classification/data

\end{thebibliography}
\end{document}
