\documentclass[landscape,a0paper,fontscale=0.285]{baposter}

\usepackage{calc}
\usepackage{graphicx}
\usepackage{amsmath}
\usepackage{amssymb}
\usepackage{relsize}
\usepackage{multirow}
\usepackage{rotating}
\usepackage{bm}
\usepackage{url}

\usepackage{multicol}

%\usepackage{times}
%\usepackage{helvet}
%\usepackage{bookman}
\usepackage{palatino}

\newcommand{\captionfont}{\footnotesize}

\graphicspath{{../images/}}
\DeclareGraphicsExtensions{.pdf,.jpg,.png}


\usetikzlibrary{calc}

\newcommand{\SET}[1]  {\ensuremath{\mathcal{#1}}}
\newcommand{\MAT}[1]  {\ensuremath{\boldsymbol{#1}}}
\newcommand{\VEC}[1]  {\ensuremath{\boldsymbol{#1}}}
\newcommand{\Video}{\SET{V}}
\newcommand{\video}{\VEC{f}}
\newcommand{\track}{x}
\newcommand{\Track}{\SET T}
\newcommand{\LMs}{\SET L}
\newcommand{\lm}{l}
\newcommand{\PosE}{\SET P}
\newcommand{\posE}{\VEC p}
\newcommand{\negE}{\VEC n}
\newcommand{\NegE}{\SET N}
\newcommand{\Occluded}{\SET O}
\newcommand{\occluded}{o}

%%%%%%%%%%%%%%%%%%%%%%%%%%%%%%%%%%%%%%%%%%%%%%%%%%%%%%%%%%%%%%%%%%%%%%%%%%%%%%%%
%%%% Some math symbols used in the text
%%%%%%%%%%%%%%%%%%%%%%%%%%%%%%%%%%%%%%%%%%%%%%%%%%%%%%%%%%%%%%%%%%%%%%%%%%%%%%%%

%%%%%%%%%%%%%%%%%%%%%%%%%%%%%%%%%%%%%%%%%%%%%%%%%%%%%%%%%%%%%%%%%%%%%%%%%%%%%%%%
% Multicol Settings
%%%%%%%%%%%%%%%%%%%%%%%%%%%%%%%%%%%%%%%%%%%%%%%%%%%%%%%%%%%%%%%%%%%%%%%%%%%%%%%%
\setlength{\columnsep}{1.5em}
\setlength{\columnseprule}{0mm}

%%%%%%%%%%%%%%%%%%%%%%%%%%%%%%%%%%%%%%%%%%%%%%%%%%%%%%%%%%%%%%%%%%%%%%%%%%%%%%%%
% Save space in lists. Use this after the opening of the list
%%%%%%%%%%%%%%%%%%%%%%%%%%%%%%%%%%%%%%%%%%%%%%%%%%%%%%%%%%%%%%%%%%%%%%%%%%%%%%%%
\newcommand{\compresslist}{%
\setlength{\itemsep}{1pt}%
\setlength{\parskip}{0pt}%
\setlength{\parsep}{0pt}%
}

%%%%%%%%%%%%%%%%%%%%%%%%%%%%%%%%%%%%%%%%%%%%%%%%%%%%%%%%%%%%%%%%%%%%%%%%%%%%%%
%%% Begin of Document
%%%%%%%%%%%%%%%%%%%%%%%%%%%%%%%%%%%%%%%%%%%%%%%%%%%%%%%%%%%%%%%%%%%%%%%%%%%%%%

\begin{document}

%%%%%%%%%%%%%%%%%%%%%%%%%%%%%%%%%%%%%%%%%%%%%%%%%%%%%%%%%%%%%%%%%%%%%%%%%%%%%%
%%% Here starts the poster
%%%---------------------------------------------------------------------------
%%% Format it to your taste with the options
%%%%%%%%%%%%%%%%%%%%%%%%%%%%%%%%%%%%%%%%%%%%%%%%%%%%%%%%%%%%%%%%%%%%%%%%%%%%%%
% Define some colors

%\definecolor{lightblue}{cmyk}{0.83,0.24,0,0.12}
\definecolor{lightblue}{rgb}{0.145,0.6666,1}

% Draw a video
\newlength{\FSZ}
\newcommand{\drawvideo}[3]{% [0 0.25 0.5 0.75 1 1.25 1.5]
  \noindent\pgfmathsetlength{\FSZ}{\linewidth/#2}
  \begin{tikzpicture}[outer sep=0pt,inner sep=0pt,x=\FSZ,y=\FSZ]
    \draw[color=lightblue!50!black] (0,0) node[outer sep=0pt, inner sep=0pt, text width=\linewidth, minimum height=0]
      (video) {\noindent#3};
    \path [fill=lightblue!50!black,line width=0pt]
      (video.north west) rectangle ([yshift=\FSZ] video.north east)
    \foreach \x in {1,2,...,#2} {
      {[rounded corners=0.6] ($(video.north west)+(-0.7,0.8)+(\x,0)$) rectangle +(0.4,-0.6)}
    }
  ;
    \path [fill=lightblue!50!black,line width=0pt]
      ([yshift=-1\FSZ] video.south west) rectangle (video.south east)
    \foreach \x in {1,2,...,#2} {
      {[rounded corners=0.6] ($(video.south west)+(-0.7,-0.2)+(\x,0)$) rectangle +(0.4,-0.6)}
    }
  ;
    \foreach \x in {1,...,#1} {
     \draw[color=lightblue!50!black] ([xshift=\x\linewidth/#1] video.north west)
                                  -- ([xshift=\x\linewidth/#1] video.south west);
    }
    \foreach \x in {0,#1} {
     \draw[color=lightblue!50!black] ([xshift=\x\linewidth/#1,yshift=1\FSZ] video.north west)
                                  -- ([xshift=\x\linewidth/#1,yshift=-1\FSZ] video.south west);
    }
  \end{tikzpicture}
}

\hyphenation{resolution occlusions}
%%
\begin{poster}%
  % Poster Options
  {
  % Show grid to help with alignment
  grid=false,
  % Column spacing
  colspacing=1em,
  % Color style
  bgColorOne=white,
  bgColorTwo=white,
  borderColor=lightblue,
  headerColorOne=black,
  headerColorTwo=lightblue,
  headerFontColor=white,
  boxColorOne=white,
  boxColorTwo=lightblue,
  % Format of textbox
  textborder=roundedleft,
  % Format of text header
  eyecatcher=true,
  headerborder=closed,
  headerheight=0.1\textheight,
%  textfont=\sc, An example of changing the text font
  headershape=roundedright,
  headershade=shadelr,
  headerfont=\Large\bf\textsc, %Sans Serif
  textfont={\setlength{\parindent}{1.5em}},
  boxshade=plain,
%  background=shade-tb,
  background=plain,
  linewidth=2pt
  }
  % Eye Catcher
  {\includegraphics[height=5em]{SU_New_BlockStree_2color}}
  % Title
  {\bf\textsc{Plant Leaf Recognition}\vspace{0.5em}}
  % Authors
  {\textsc{\{ Albert Liu and Yangming Wang \}@stanford.edu}}
  % University logo
  {% The makebox allows the title to flow into the logo, this is a hack because of the L shaped logo.
    \includegraphics[height=5.0em]{SU_Seal_Red}
  }

%%%%%%%%%%%%%%%%%%%%%%%%%%%%%%%%%%%%%%%%%%%%%%%%%%%%%%%%%%%%%%%%%%%%%%%%%%%%%%
%%% Now define the boxes that make up the poster
%%%---------------------------------------------------------------------------
%%% Each box has a name and can be placed absolutely or relatively.
%%% The only inconvenience is that you can only specify a relative position
%%% towards an already declared box. So if you have a box attached to the
%%% bottom, one to the top and a third one which should be in between, you
%%% have to specify the top and bottom boxes before you specify the middle
%%% box.
%%%%%%%%%%%%%%%%%%%%%%%%%%%%%%%%%%%%%%%%%%%%%%%%%%%%%%%%%%%%%%%%%%%%%%%%%%%%%%
    %
    % A coloured circle useful as a bullet with an adjustably strong filling
    \newcommand{\colouredcircle}{%
      \tikz{\useasboundingbox (-0.2em,-0.32em) rectangle(0.2em,0.32em);
            \draw[draw=black,fill=lightblue,line width=0.03em] (0,0) circle(0.18em);
      }
    }

%%%%%%%%%%%%%%%%%%%%%%%%%%%%%%%%%%%%%%%%%%%%%%%%%%%%%%%%%%%%%%%%%%%%%%%%%%%%%%
  \headerbox{Problem}{name=problem,column=0,row=0}{
%%%%%%%%%%%%%%%%%%%%%%%%%%%%%%%%%%%%%%%%%%%%%%%%%%%%%%%%%%%%%%%%%%%%%%%%%%%%%%
Fine-grained leaf recognition has important application in weeds identification, species discovery, plant taxonomy, etc.
However, the subtle differences between species and the sheer number of categories makes it hard to solve.

\centering\includegraphics[width=0.65\linewidth]{easilyconfused}

%Traditionally, numerous hand-crafted features have been proposed and recently ConvNets based approaches are applied to
% this problem.
   \vspace{0.3em}
 }

%%%%%%%%%%%%%%%%%%%%%%%%%%%%%%%%%%%%%%%%%%%%%%%%%%%%%%%%%%%%%%%%%%%%%%%%%%%%%%
  \headerbox{Method}{name=method,column=0, below=problem, above=bottom}{
%%%%%%%%%%%%%%%%%%%%%%%%%%%%%%%%%%%%%%%%%%%%%%%%%%%%%%%%%%%%%%%%%%%%%%%%%%%%%%
  {\centering\includegraphics[width=1.00\linewidth]{overview}}
\begin{enumerate}\compresslist
  \item Preprocessing

    Apply CLAHE to reduce lighting condition variation and resize to fit the next layer. Selectively use K-means to remove background heuristically. For challenging dataset, find the convex hull containing the largest N contours and then use GrabCut to segment leaf out.
  \item Feature extraction
    \begin{itemize}
     \item ConvNets.

      Transfer learning approach. Specifically, we take a couple of ConvNets pretrained on ImageNet for ILSVRC object classification task, remove top layers and use it as generic feature extractor.
     \item Traditional SIFT + BoF

      Key points are densely sampled. Size of the codebook (K) is fixed at 1000/3000.
    \end{itemize}
  \item Classification
    SVM Linear/RBF/Softmax/MLP/etc.
\end{enumerate}
{\centering\includegraphics[width=1.00\linewidth]{bof}}
  }



%%%%%%%%%%%%%%%%%%%%%%%%%%%%%%%%%%%%%%%%%%%%%%%%%%%%%%%%%%%%%%%%%%%%%%%%%%%%%%
\headerbox{Experimental Results}{name=results,column=1,span=2,row=0}{
  %%%%%%%%%%%%%%%%%%%%%%%%%%%%%%%%%%%%%%%%%%%%%%%%%%%%%%%%%%%%%%%%%%%%%%%%%%%%%%
  %\begin{tabular}{c@{}c@{}}
  $\vcenter{\hbox{\includegraphics[width=0.45\linewidth]{test_accuracy}}}$
  $\vcenter{\hbox{\includegraphics[width=0.50\linewidth]{best_cm_uniform}}}$
  %{\includegraphics[width=0.55\linewidth]{best_cm_uniform}}
  %\end{tabular}
}


%%%%%%%%%%%%%%%%%%%%%%%%%%%%%%%%%%%%%%%%%%%%%%%%%%%%%%%%%%%%%%%%%%%%%%%%%%%%%%
  \headerbox{Transfer Learning - ConvNet}{name=convnet,column=1,span=2, below=results, above=bottom}{
%%%%%%%%%%%%%%%%%%%%%%%%%%%%%%%%%%%%%%%%%%%%%%%%%%%%%%%%%%%%%%%%%%%%%%%%%%%%%%
  The pre-trained weights of VGG16, VGG19 and ResNet50 are available from open source Keras Framework.
  After comparing preliminary results, we choose ResNet50 since ResNet50 gives better results and less overfitting. We believe this can be attributed to the fact that ResNet50 is deeper and generates lower dimension feature vector, which is likely due to the use of a more aggressive Average Pooling with a pool size of 7x7.
  %\vspace{0.3em}
  {\includegraphics[width=1\linewidth]{transfer_learning}}
  The ResNet is famous for it's deep layers, in our case, 50 layers, with 49 Conv layers and one FC layer
  on top. Except for the first Conv layer, the rest 48 composes 16 ``residual'' blocks in 4 stages. The block within
  each stage has similar architecture, i.e. same input \& output shape. \par

  The output of the stage 5 give 2048-D features. Every cell in the grid shown above is a 8x8 filter visualized with
  heatmap, which will be a scalar after average pooling.
}

%%%%%%%%%%%%%%%%%%%%%%%%%%%%%%%%%%%%%%%%%%%%%%%%%%%%%%%%%%%%%%%%%%%%%%%%%%%%%%
\headerbox{DATASET}{name=dataset,column=3, row=0}{
%%%%%%%%%%%%%%%%%%%%%%%%%%%%%%%%%%%%%%%%%%%%%%%%%%%%%%%%%%%%%%%%%%%%%%%%%%%%%%
\begin{enumerate}\compresslist
\item Swedish/Flavia Leaf Dataset: clean images taken in controlled conditions. Use 20 samples per species for training.
\item ImageCLEF Plant: crowd sourced and noisy. Considerable variations on lighting conditions, viewpoints, background
      clutters and occlusions. Choose species with at least 20 training samples.
  \begin{tabular}{c@{}c@{}c@{}c@{}c@{}}
  {\includegraphics[height=4.0em]{swedish1}} &
  {\includegraphics[height=4.0em]{flavia1}} &
  {\includegraphics[height=4.0em]{clef1}} &
  {\includegraphics[height=4.0em]{clef2}} &
  {\includegraphics[height=4.0em]{clef3}}
  \end{tabular}
\end{enumerate}
}

%%%%%%%%%%%%%%%%%%%%%%%%%%%%%%%%%%%%%%%%%%%%%%%%%%%%%%%%%%%%%%%%%%%%%%%%%%%%%%
\headerbox{Discussions}{name=discussion,column=3,span=1, below=dataset}{
  %%%%%%%%%%%%%%%%%%%%%%%%%%%%%%%%%%%%%%%%%%%%%%%%%%%%%%%%%%%%%%%%%%%%%%%%%%%%%%
  \begin{enumerate}
    \item As expected, CNN codes off the shelf yields similar or better accuracy, compared to SIFT+BoF. Particularly traditional method suffers with noisy datasets.
    \item Error analysis shows that it helps greatly to reduce noise/variations on data.
    \item Looking at the confusion matrix, we believe the main causes for misclassification
      \begin{itemize}
        \item [--] Very fine differences between species, which is hard even for human experts
        \item [--] Noisy and possibly non-representative train data lead to overfitting,
        \item [--] Since the ConvNet models are pre-trained for a different task, we speculate these features may not always generalize well.
      \end{itemize}
  \end{enumerate}
   \vspace{0.3em}
}

%%%%%%%%%%%%%%%%%%%%%%%%%%%%%%%%%%%%%%%%%%%%%%%%%%%%%%%%%%%%%%%%%%%%%%%%%%%%%%
\headerbox{Future works}{name=future,column=3, span=1, above=bottom, below=discussion}{
  %%%%%%%%%%%%%%%%%%%%%%%%%%%%%%%%%%%%%%%%%%%%%%%%%%%%%%%%%%%%%%%%%%%%%%%%%%%%%%
  \begin{enumerate}
    \item Acquire more data and fine-tune ConvNet to solve overfitting problem
    \item Engage advanced techniques for image augmentation
    \item Explore state-of-art method to detect and locate leaf for Image CLEF natural leaf dataset.
  \end{enumerate}
}

% %%%%%%%%%%%%%%%%%%%%%%%%%%%%%%%%%%%%%%%%%%%%%%%%%%%%%%%%%%%%%%%%%%%%%%%%%%%%%%
%   \headerbox{References}{name=references,column=1, span=1, above=bottom}{
% %%%%%%%%%%%%%%%%%%%%%%%%%%%%%%%%%%%%%%%%%%%%%%%%%%%%%%%%%%%%%%%%%%%%%%%%%%%%%%
%     \smaller
%     \bibliographystyle{ieee}
%     \renewcommand{\section}[2]{\vskip 0.05em}
%       \begin{thebibliography}{1}\itemsep=-0.01em
%       \setlength{\baselineskip}{0.4em}
%       \bibitem{Charles13}
%   Charles Mallah, James Cope, James Orwell.
%   \newblock Plant Leaf Classification Using Probabilistic Integration of Shape, Texture and Margin Features. Signal
%           Processing, Pattern Recognition and Applications, in press. 2013
%       \end{thebibliography}
%   }

\end{poster}

\end{document}
